\section{\textit{mtplottools} }
\label{sec:imaging.mtplottools}

A collection of functions taht enable the plotting and/or visualisation of data for all MT analysis steps. 

\subsection{\textit{plotcoh} (Function)}
\label{ssec:imaging.mtplottools.plotcoh}

plot coherence output from birrp\_bbconvert.  If you want to save the
    plot do so using the interactive gui.

\subsection{\textit{plotResPhase} (Function)}
\label{ssec:imaging.mtplottools.plotResPhase}

plot the apparent resistivity and 
    phase for TE and TM modes 
    from a .dat file produced by writedat.  If you want to save the plot 
    use the save button on top left.

\subsection{\textit{resPhasePlots} (Function)}
\label{ssec:imaging.mtplottools.resPhasePlots}

plot multiple responses given full path filenames.  
    Can input key word list dictionary if parameters for each plot are different
    or just single values.

\subsection{\textit{plotTSplotTipper} (Function)}
\label{ssec:imaging.plotTSmtplottools.plotTipper}

plot the real and imaginary induction arrow from EDI file.

ToDo !!! void function !!!

\subsection{\textit{plotTS} (Function)}
\label{ssec:imaging.mtplottools.plotTS}

plot timeseries that have been combined using combineFewFiles.
    combinefilelst is the output list of filenames

\subsection{\textit{plotPTpseudoSection} (Function)}
\label{ssec:imaging.mtplottools.plotPTpseudoSection}

plot a pseudo section of phase tensor ellipses given a list of 
    full path filenames.

\subsection{\textit{plotRTpseudoSection} (Function)}
\label{ssec:imaging.mtplottools.plotRTpseudoSection}

plot a pseudo section of resistivity tensor ellipses given a list of 
    full path filenames. (Weckmann et al. 2002)

\subsection{\textit{plotPTMaps} (Function)}
\label{ssec:imaging.mtplottools.plotPTMaps}

Plots phase tensor ellipses in map view from a list of edifiles with full 
    path.

\subsection{\textit{plotResPhasePseudoSection} (Function)}
\label{ssec:imaging.mtplottools.plotResPhasePseudoSection}

plots a  pseudo section from a list of edifiles with full path with descending 
    frequency or ascending period on the y axis and relative position on the x.

\subsection{\textit{plotRTMaps} (Function)}
\label{ssec:imaging.mtplottools.plotRTMaps}

plots phase tensor ellipses in map view from a list of edifiles with full 
    path.

\subsection{\textit{plotRoseStrikeAngles} (Function)}
\label{ssec:imaging.mtplottools.plotRoseStrikeAngles}


plots the strike angle as determined by phase tensor azimuth (Caldwell et 
    al. [2004]) and invariants of the impedance tensor (Weaver et al. [2003]).
    
    The data is split into decades where the histogram for each is plotted in 
    the form of a rose diagram with a range of 0 to 180 degrees.
    Where 0 is North and 90 is East.   The median angle of the period band is 
    set in polar diagram.  The top row is the strike estimated from
    the invariants of the impedance tensor.  The bottom row is the azimuth
    estimated from the phase tensor.  If tipper is 'y' then the 3rd row is the
    strike determined from the tipper, which is orthogonal to the induction
    arrow direction.  


\section{\textit{ptplots}}
\label{sec:imaging.ptplots}

Ploitting functions and routines in connection to the phase tensor

\subsection{\textit{ptplots} (Class)}
\label{ssec:imaging.ptplots.ptplots}

Generate a PhaseTensorPlot object. Various visualisations can be done from this.

   Plots are: 
   \begin{itemize}
    
    \item  plotPhaseTensor(save='n',fmt='pdf',fignum=1,thetar=0) {~}\\ 
    plots the phase tensor as 
     ellipses with long axis directed towards electrical strike and short axis
     directed across magnetic strike.  Spaces the ellipses across the period axis.
    
     \item plotStrikeAngle(save='n',fmt='pdf',fignum=1,thetar=0) {~}\\-plots the electric 
     strike angle as a function of period with error bars. 
    
     \item plotMinMaxPhase(save='n',fmt='pdf',fignum=1,thetar=0) {~}\\ plots the min and max 
     phase angle as a function of period with error bars.  
    
     \item plotAzimuth(save='n',fmt='pdf',fignum=1,thetar=0) {~}\\ plots the azimuth angle as
     a function of period with error bars. 
    
     \item plotSkew(save='n',fmt='pdf',fignum=1,thetar=0) {~}\\ plots the skew angle as a 
     function of period with error bars.  
    
     \item plotElliticity(save='n',fmt='pdf',fignum=1,thetar=0) {~}\\ plots the ellipticity 
     as a function of period with error bars. 
    
     \item plotAll(save='n',fmt='pdf',fignum=1,thetar=0) {~}\\plots phase tensor, strike angle, 
     min and max phase angle, azimuth, skew, and ellipticity as subplots on one
     plot. 
   \end{itemize}

    You can save the plots by making save='y' or a path you input. fmt is the
    fmt you want to save the figure as. Can be: pdf,eps,ps,png or svg. Fignum
    is the figure number in the event you want to plot multipl plots."""



\subsubsection{\textit{plotPhaseTensor} (Method)}
\label{sssec:imaging.ptplots.ptplots.plotPhaseTensor}

plot phase tensor ellipses

\subsubsection{\textit{plotStrikeangle} (Method)}
\label{sssec:imaging.ptplots.ptplots.plotStrikeangle}

Plot the strike angle as calculated from the invariants

\subsubsection{\textit{plotMinMaxPhase} (Method)}
\label{sssec:imaging.ptplots.ptplots.plotMinMaxPhase}

Plot the minimum and maximum phase of phase tensor

\subsubsection{\textit{plotAzimuth} (Method)}
\label{sssec:imaging.ptplots.ptplots.plotAzimuth}

plot the azimuth of maximum phase

\subsubsection{\textit{plotSkew} (Method)}
\label{sssec:imaging.ptplots.ptplots.plotSkew}

Plot the skew angle

\subsubsection{\textit{plotEllipticity} (Method)}
\label{sssec:imaging.ptplots.ptplots.plotEllipticity}

Plot the ellipticity as $\frac{\Phi_{max} - \Phi_{min} }{\Phi_{max} + \Phi_{min}}$

\subsubsection{\textit{plotAll} (Method)}
\label{sssec:imaging.ptplots.ptplots.plotAll}

 plot phase tensor, strike angle, min and max phase angle, 
        azimuth, skew, and ellipticity as subplots on one plot

\subsubsection{\textit{plotResPhase} (Method)}
\label{sssec:imaging.ptplots.ptplots.plotResPhase}

plot the resistivity and phase for
        all impedance tensor polarizations.  2 plots, one containing $xy,yx$- 
        polarizations, the other $xx,yy$

